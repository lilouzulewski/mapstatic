\documentclass{beamer}
\usepackage[utf8]{inputenc}
\usetheme{Hannover}
\usecolortheme{rose}


\title[Développement Logiciel]{MAPSTATIC}
\author[]{Rim Alhajal, Thiziri Abchiche, Maryem El Yamani, Lilou Zulewski}
\institute[M1 - SSD]{M1 - SSD}
\date{November 2022}

\AtBeginSection[]
{
\begin{frame}{Plan}
\tableofcontents[currentsection]
\end{frame}
}

\begin{document}

\maketitle

\section{Données et Objectif}

\section{Visualisation}

\section{Prédiction}

\begin{frame}
\frametitle{Données et Objectif}
\begin{itemize}
    \item Créer une carte interactive de la France pour mieux visualiser les données de consommation électrique annuelle moyenne par ville.\\
    \item Prédire la consommation de différentes sources d'énergie pour le 8 Décembre 2022 par quart d'heure.
\end{itemize}
\end{frame}

\begin{frame}
\frametitle{Visualisation}
Deux différents codes ont étaient faits. Une par commune et une autre par département.\\
\end{frame}

\begin{frame}
\frametitle{Visualisation par Commune}
\begin{itemize}
    \item La moyenne des dernières années a été calculée pour chaque commune.\\
    \item Un fichier json portant les coordonnées de toutes les communes en France a été utilisé.
    \item Pour la partie interactive, au-dessus de chaque commune, une bulle apparaît montrant le nom de la commune avec la valeur exacte.
\end{itemize}
\end{frame}

\begin{frame}  
\frametitle{Visualisation par Commune}
\begin{center}
\includegraphics[width=100mm]{Screenshot (3).png}
\end{center}
\end{frame}

\begin{frame}
\frametitle{Visualisation par Département} 
\end{frame}

\begin{frame}
\frametitle{Prédiction}
\begin{itemize}
    \item En utilisant la moyenne et l'écart-type.
    \item En utilisant la méthode de lissage.
\end{itemize}
\end{frame}

\end{document}