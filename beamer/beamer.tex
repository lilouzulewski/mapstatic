\documentclass{beamer}
\usepackage[utf8]{inputenc}
\usetheme{Madrid}
\usecolortheme{rose}
\setbeamertemplate{navigation symbols}{}



\title[Développement Logiciel]{MAPSTATIC}
\author[]{Rim Alhajal, Thiziri Abchiche, Maryem El Yamani, Lilou Zulewski}
\institute{M1 - SSD}
\date{2 décembre 2022}



\begin{document}
\maketitle



\section{Données et Objectif}

\begin{frame}
\frametitle{Données et Objectif}
\begin{block}{Visualisation}
Créer une carte interactive de la France pour mieux visualiser les données de consommation électrique annuelle moyenne par ville
\end{block}
\begin{block}{Prédiction}
Prédire la consommation de différentes sources d'énergie pour le 8 Décembre 2022 par quart d'heure
\end{block}
\end{frame}



\section{Visualisation}

\begin{frame}
\frametitle{Visualisation - Objectifs}
\begin{block}{Différentes visualisations}
\begin{itemize}
    \item par communes \\
    \item par départements \\
\end{itemize}
\end{block}
\begin{block}{Traitement des données de consommation}
\begin{itemize}
    \item déduire les villes plus ou moins consommatrices de la région
\end{itemize}
\end{block}
\end{frame}

\begin{frame}
\frametitle{Visualisation par Commune}
\begin{block}{Utilisation des données et Création de la carte}
\begin{itemize}
    \item calcul de la moyenne des consommations des dernières années par commune \\
    \item utilisation d'un fichier json portant les coordonnées de toutes les communes en France
\end{itemize}
\end{block}
\begin{block}{Partie interactive}
\begin{itemize}
    \item mise en place d'une bulle contenant le nom de la commune avec la valeur exacte au-dessus de chaque commune
\end{itemize}
\end{block}
\end{frame}

\begin{frame}
\frametitle{Visualisation par Département}
\begin{block}{Utilisation des données}
\begin{itemize}
    \item modification des données pour en extraire les données par département
\end{itemize}
\end{block}
\begin{block}{Création de la Carte}
\begin{itemize}
    \item utilisation du geojson et choropleth
\end{itemize}
\end{block}
\end{frame}

\begin{frame}
\frametitle{Traitement des Données de Consommation}
\begin{block}{Connaître les communes impactantes sur la consommation}
\begin{itemize}
    \item  histogramme des moyennes des consommations annuelles de chaque commune de la région
    \item  histogramme des consommations des villes les plus et les moins consommatrices
\end{itemize}
\end{block}
\end{frame}



\section{Prédiction}
\begin{frame}{Prédiction - Récupération des Données}
\begin{block}{Importation des Données}
\begin{itemize}
    \item création d'une classe
\end{itemize}
\end{block}
\begin{block}{Mise en Forme des Données}
\begin{itemize}
    \item création de nouveaux jeux de données pour chaque énergie
    \item suppression des données manquantes
\end{itemize}
\end{block}
\end{frame}

\begin{frame}
\frametitle{Prédiction - Méthodes}
\begin{block}{Différentes Méthodes de Prédiction}
  \begin{itemize}
    \item la moyenne et l'écart-type
    \item un modèle ARIMA
    \item un modèle PROPHET
\end{itemize} 
\end{block}
\end{frame}



\section{Git}
\begin{frame}{Mise en Forme du Git}
\begin{block}{Fichiers Supplémentaires}
    \begin{itemize}
    \item init
    \item setup
    \item workflow
    \item requirements
\end{itemize}
\end{block}
\begin{block}{Documentation}
\begin{itemize}
    \item tutoriel d'importation du module mapstatic
    \item site web
\end{itemize}
\end{block}
\end{frame}



\end{document}
